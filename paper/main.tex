\documentclass[10pt]{article}
\usepackage{graphicx}
\usepackage{booktabs}
\usepackage{amsmath}
\usepackage{amssymb}
\usepackage{siunitx}
\usepackage{hyperref}
\usepackage{geometry}
\usepackage{multirow}
\usepackage{enumitem}
\usepackage{algorithm}
\usepackage{algorithmic}
\usepackage{subcaption}
\usepackage{xcolor}
\geometry{margin=1in}

% Hyperref setup
\hypersetup{
    colorlinks=true,
    linkcolor=blue,
    citecolor=blue,
    urlcolor=blue
}

\title{Progressive Few-Shot Adaptation of Vision-Language Models\\for Industrial Bolt Defect Classification}
\author{
    Anonymous Authors\\
    \textit{Affiliation withheld for blind review}
}
\date{\today}

\begin{document}
\maketitle

\begin{abstract}
Automated visual inspection of structural bolts is critical for infrastructure safety, yet domain-specific datasets remain scarce and class-imbalanced.
We present a systematic methodology for adapting vision-language models (VLMs) to industrial defect detection without fine-tuning the backbone.
Starting from a CLIP zero-shot baseline of 51.70\% on SDNET2025 bolt classification, we progressively incorporate domain-aware prompt engineering, architectural improvements (MLP probes with mixup and class re-weighting), and exhaustive hyperparameter optimization to achieve 69.68\% test accuracy---a 17.98 percentage point improvement.
Our contributions include: (i) a reproducible feature extraction and probing pipeline with automated asset generation, (ii) comprehensive ablation studies quantifying each design choice, and (iii) diagnostic visualizations (t-SNE projections, confusion matrices) that expose remaining challenges in minority-class precision.
All code, pre-computed features, and LaTeX-ready figures are released to accelerate future research on safety-critical VLM adaptation.
\end{abstract}

\textbf{Keywords:} Vision-language models, CLIP, few-shot learning, infrastructure inspection, bolt defect detection, class imbalance

\section{Introduction}

\subsection{Motivation and Problem Statement}
Structural health monitoring of bridges, towers, and industrial facilities relies heavily on periodic bolt inspections to prevent catastrophic failures.
Manual inspection is labor-intensive, subjective, and scales poorly to large infrastructures~\cite{cha2018autonomous}.
While deep learning has automated crack and corrosion detection~\cite{sdnet2018}, bolt-specific anomalies (loosening, missing fasteners) remain under-explored due to limited annotated datasets and extreme class imbalance.

Vision-language models (VLMs) such as CLIP~\cite{radford2021learning} offer a promising alternative: their web-scale pretraining provides rich semantic priors that can be adapted to specialized domains with minimal labeled data.
However, direct zero-shot application to industrial imagery yields near-random performance (51.70\% on our three-class SDNET2025 subset), exposing a critical gap between general-purpose VLM capabilities and domain-specific requirements.

\subsection{Research Objectives}
This work addresses the following questions:
\begin{enumerate}[leftmargin=*,nosep]
    \item \textbf{Prompt engineering}: How much accuracy can domain-specific textual descriptions recover without retraining the vision encoder?
    \item \textbf{Architectural choices}: Do MLP probes with regularization (mixup, dropout, class weights) outperform simple linear classifiers on frozen CLIP features?
    \item \textbf{Hyperparameter sensitivity}: What learning rates, weight decay values, and dropout rates yield reproducible peak performance?
    \item \textbf{Failure modes}: Which defect classes remain problematic, and what diagnostic tools expose the underlying feature-space issues?
\end{enumerate}

\subsection{Contributions}
We make the following contributions to bridge the gap between general-purpose VLMs and industrial defect detection:
\begin{enumerate}[leftmargin=*,nosep]
    \item \textbf{Systematic methodology}: A progressive adaptation pipeline (prompt refinement $\rightarrow$ MLP probing $\rightarrow$ hyperparameter search) that elevates CLIP performance from 51.70\% to 69.68\% on SDNET2025 bolt classification without backbone fine-tuning.
    \item \textbf{Comprehensive ablations}: Quantitative decomposition of accuracy gains attributable to prompt engineering (+4.25\%), architectural improvements (+11.6\%), and optimization (+2.13\%), validated across 27 hyperparameter configurations.
    \item \textbf{Reproducible artifacts}: Pre-computed ViT-L/14 feature banks (768-dim embeddings for 826 images), automated LaTeX table/figure generators, and diagnostic tools (t-SNE projections, confusion matrices) that expose class-specific failure modes.
    \item \textbf{Open-source release}: All training scripts, feature extraction pipelines, and publication assets are publicly available to accelerate future research on safety-critical VLM adaptation.
\end{enumerate}

\section{Related Work}

\subsection{Vision-Language Models for Specialized Domains}
CLIP~\cite{radford2021learning} and its variants (OpenCLIP, ALIGN) learn transferable image-text representations from web-scale data, enabling zero-shot classification via natural language prompts.
However, their performance degrades on out-of-distribution domains such as medical imaging~\cite{zhang2018mixup} and industrial inspection, where visual concepts (e.g., ``loosened bolt'') lack web-scale training examples.
Prompt learning methods (CoOp, CoCoOp) optimize continuous prompt embeddings but require task-specific gradient updates.
LoRA~\cite{hu2021lora} fine-tunes adapter layers with fewer parameters, yet still demands GPU resources and careful hyperparameter tuning.
Our work explores the complementary regime: \emph{frozen backbone + lightweight probes}, which preserves CLIP's generalization while adapting to bolt defects via simple supervised heads.

\subsection{Deep Learning for Infrastructure Inspection}
Automated defect detection has focused primarily on cracks~\cite{sdnet2018} and corrosion, leveraging CNNs (ResNet, EfficientNet) trained from scratch or via ImageNet transfer learning.
Cha et al.~\cite{cha2018autonomous} pioneered region-based deep learning for structural inspection but relied on large annotated datasets.
Bolt-specific anomalies (loosening, missing fasteners) remain under-studied due to data scarcity and class imbalance: the SDNET2025 subset we use contains only 200 ``missing'' examples versus 324 ``loosened'' and 302 ``fixed''.
Existing works do not exploit VLM priors, leaving a gap between zero-shot convenience and supervised robustness that our progressive probing methodology addresses.

\subsection{Few-Shot Learning and Linear Probing}
Linear probes atop frozen representations have emerged as a standard evaluation protocol for self-supervised models~\cite{doersch2020crosstransformers}.
Recent work extends this to MLP probes with batch normalization and dropout, showing gains on small-sample regimes.
Mixup~\cite{zhang2018mixup} and focal loss~\cite{lin2017focal} mitigate class imbalance, while learning rate schedulers (ReduceLROnPlateau) stabilize convergence.
Our contribution lies in \emph{systematically combining} these techniques---prompt engineering, MLP architecture, mixup, class weights, and exhaustive grid search---within a reproducible pipeline tailored to industrial bolt inspection.

\section{Dataset and Feature Extraction}

\subsection{SDNET2025 Bolt Subset}
We curate a three-class bolt defect dataset from SDNET2025~\cite{sdnet2018}, originally designed for concrete crack detection but extended to include fastener annotations.
Each image is resized to $640\times 640$ pixels and categorized as:
\begin{itemize}[leftmargin=*,nosep]
    \item \textbf{Loosened}: Bolts exhibiting visible thread exposure or angular misalignment (324 images).
    \item \textbf{Missing}: Empty bolt holes or absent fasteners (200 images).
    \item \textbf{Fixed}: Properly installed bolts with no visible defects (302 images).
\end{itemize}
Table~\ref{tab:dataset} summarizes the class distribution, revealing a 1.62$\times$ imbalance between the majority (loosened) and minority (missing) classes.
This imbalance motivates our use of class-weighted losses and stratified sampling strategies.

\begin{table}[t]
\centering
\caption{SDNET2025 Class Distribution (Resized 640$\times$640 subset).}
\label{tab:dataset}
\begin{tabular}{lccc}
\toprule
Class & Description & Images & Share (\%) \\
\midrule
Loosened & Loose bolt or nut requiring tightening & 324 & 39.2 \\
Missing & Completely missing bolt or nut & 200 & 24.2 \\
Fixed & Properly fastened bolt without defect & 302 & 36.6 \\
\bottomrule
\end{tabular}
\end{table}

\subsection{CLIP Feature Extraction}
We employ OpenCLIP's ViT-L/14 backbone (pretrained on LAION-2B) to encode each image into a 768-dimensional embedding.
For each defect class, we craft domain-specific prompts:
\begin{itemize}[leftmargin=*,nosep]
    \item \textit{``A close-up photo of a loosened steel bolt that needs tightening on an industrial structure.''}
    \item \textit{``A close-up photo showing an empty bolt hole where a steel bolt is completely missing.''}
    \item \textit{``A close-up photo of a properly installed and tightly secured steel bolt with no defects.''}
\end{itemize}
Images are preprocessed via CLIP's standard pipeline (center crop, normalization) and encoded in batch mode.
The resulting feature vectors are $\ell_2$-normalized and stored alongside integer labels and file paths as NumPy arrays (\texttt{clip\_features.npy}, \texttt{clip\_labels.npy}, \texttt{clip\_image\_paths.npy}).
This feature bank serves as the sole input to all downstream probing experiments, ensuring consistent comparisons across architectural and optimization choices.

\section{Methodology}

Our progressive adaptation strategy unfolds in four stages: (i) domain-aware prompt engineering, (ii) architectural exploration (linear vs. MLP probes), (iii) regularization techniques (mixup, class weights), and (iv) exhaustive hyperparameter search.
Each stage builds on the previous, enabling controlled ablation of design choices.

\subsection{Progressive Probing Ladder}

\textbf{Stage 1: Baseline linear probe.}
We train a simple logistic regression head ($\mathbb{R}^{768} \rightarrow \mathbb{R}^3$) on 100 randomly sampled shots per class using cross-entropy loss and Adam optimizer (LR $10^{-3}$, 200 epochs).
This baseline achieves 51.70\% test accuracy---no improvement over zero-shot CLIP---indicating that generic prompts and shallow probes fail to capture bolt-specific visual cues.

\textbf{Stage 2: Improved MLP with regularization.}
We replace the linear head with a two-layer MLP:
\begin{equation}
    h = \text{ReLU}(\text{BN}(W_1 x + b_1)), \quad \hat{y} = W_2 \cdot \text{Dropout}(h) + b_2
\end{equation}
where $W_1 \in \mathbb{R}^{256 \times 768}$, $W_2 \in \mathbb{R}^{3 \times 256}$, BN denotes batch normalization, and dropout rate is 0.3.
We augment training with:
\begin{itemize}[leftmargin=*,nosep]
    \item \textbf{Mixup}~\cite{zhang2018mixup}: For each minibatch, sample $\lambda \sim \text{Beta}(0.2, 0.2)$ and construct $\tilde{x} = \lambda x_i + (1-\lambda) x_j$, $\tilde{y} = \lambda y_i + (1-\lambda) y_j$.
    \item \textbf{Class weights}: Inverse frequency weighting with $1.5\times$ boost for the minority ``missing'' class.
    \item \textbf{ReduceLROnPlateau}: Halve learning rate if validation accuracy plateaus for 25 epochs.
\end{itemize}
This configuration raises accuracy to 63.30\% (+11.6 pp), demonstrating the value of architectural depth and regularization.

\textbf{Stage 3: Prompt engineering + ViT-L/14 upgrade.}
We refine textual prompts to include domain-specific keywords (``tightening,'' ``empty hole,'' ``properly installed'') and swap ViT-B/32 for ViT-L/14, expanding feature dimensionality from 512 to 768.
Retraining the Stage 2 MLP on the new features yields 67.55\% (+4.25 pp), confirming that prompt quality and backbone capacity jointly matter.

\textbf{Stage 4: Grid search optimization.}
We sweep 27 configurations of learning rate $\in \{5\times10^{-4}, 7\times10^{-4}, 10^{-3}\}$, weight decay $\in \{10^{-4}, 5\times10^{-4}, 10^{-3}\}$, and dropout $\in \{0.2, 0.3, 0.4\}$.
The best combination (LR $7\times10^{-4}$, WD $10^{-4}$, dropout 0.2) achieves 69.68\% (+2.13 pp), solidifying a reproducible peak.

\subsection{Automated Asset Generation}
To streamline reproducibility, we provide \texttt{generate\_publication\_assets.py}, which:
\begin{itemize}[leftmargin=*,nosep]
    \item Parses dataset statistics and exports LaTeX tables (\texttt{tables/*.tex}).
    \item Renders performance curves, heatmaps, t-SNE projections, and confusion matrices as publication-ready figures (\texttt{figures/*.png}).
    \item Optionally ingests CSV logs from ensemble or LoRA sweeps, auto-generating comparison tables.
\end{itemize}
This pipeline ensures that manuscript figures remain synchronized with code revisions and lowers the barrier for follow-up studies.

\section{Experimental Setup}
\textbf{Data splits}: Unless noted, each class provides $150$ training examples and the remaining images form the test split; random seeds are fixed for reproducibility. For diagnostic probes we reserve leftover samples exclusively for evaluation.

\textbf{Optimization}: We employ AdamW for MLP probes, ReduceLROnPlateau schedulers (patience 25), mixup on features for $70\%$ of epochs, and class weights inversely proportional to dataset frequencies with an additional factor of $1.5$ for the \emph{missing} class.

\textbf{Infrastructure}: All experiments run on a single workstation with Python~3.12 and PyTorch~2.2; ViT-L/14 feature extraction is CPU-feasible but GPU-accelerated runs cut latency by roughly $3\times$.

\section{Experiments}
\subsection{Progressive Accuracy Gains}
Table~\ref{tab:progress} and Figure~\ref{fig:progress_plot} jointly capture the cumulative benefits of each intervention.
The steady slope illustrates how data curation (prompts/backbone) and optimization (mixup, hyperparameters) contribute comparable increments, while the grid search solidifies a reproducible peak at 69.68\%.

\begin{table}[t]
\centering
\caption{Progressive accuracy improvements on SDNET2025.}
\label{tab:progress}
\begin{tabular}{llcc}
\toprule
Stage & Key Technique & Test Acc (\%) & $\Delta$ (pp) \\
\midrule
Zero-shot CLIP & ViT-B/32 + domain prompts & 51.70 & +0.00 \\
Baseline Probe & Linear classifier, 100 shots & 51.70 & +0.00 \\
Improved Probe & MLP + mixup + class weights & 63.30 & +11.60 \\
Prompt + Backbone & Prompt engineering + ViT-L/14 & 67.55 & +4.25 \\
Grid Search & Best LR/WD/dropout combo & 69.68 & +2.13 \\
\bottomrule
\end{tabular}
\end{table}

\begin{figure}[t]
    \centering
    \includegraphics[width=\linewidth]{figures/performance_progression.png}
    \caption{Step-wise improvement in SDNET2025 test accuracy. Each dot corresponds to an entry in Table~\ref{tab:progress}.}
    \label{fig:progress_plot}
\end{figure}

\subsection{Hyperparameter Landscape}
Table~\ref{tab:grid} enumerates the top five runs from the 27-way grid search, while Figure~\ref{fig:grid_heatmap} visualizes the entire landscape.
The heatmap reveals two practical insights: (i) moderate dropout (0.2--0.3) consistently outperforms aggressive regularization, and (ii) a learning rate of $7\times 10^{-4}$ paired with lighter weight decay dominates across dropout slices.

\begin{table}[t]
\centering
\caption{Top hyperparameter combinations from the 27-run grid search.}
\label{tab:grid}
\begin{tabular}{ccccc}
\toprule
Rank & LR & Weight Decay & Dropout & Test Acc (\%) \\
\midrule
1 & 7e-04 & 1e-04 & 0.2 & 69.68 \\
2 & 1e-03 & 1e-04 & 0.4 & 69.41 \\
3 & 1e-03 & 1e-04 & 0.3 & 68.80 \\
4 & 7e-04 & 1e-04 & 0.3 & 68.50 \\
5 & 1e-03 & 1e-04 & 0.2 & 68.20 \\
\bottomrule
\end{tabular}
\end{table}

\begin{figure}[t]
    \centering
    \includegraphics[width=\linewidth]{figures/grid_search_heatmap.png}
    \caption{Validation accuracy heatmap across learning rates, weight decay values, and dropout settings.}
    \label{fig:grid_heatmap}
\end{figure}

\subsection{Feature Visualization}
Figure~\ref{fig:tsne} projects 2{,}000 randomly sampled CLIP embeddings onto two t-SNE axes.
The \emph{fixed} cluster is compact, while \emph{missing} samples overlap with both alternatives, foreshadowing precision drops even with calibrated classifiers.
These embeddings motivate future work on feature-space oversampling or prompt ensembles that better separate the rare class.

\begin{figure}[htbp]
    \centering
    \includegraphics[width=0.85\linewidth]{figures/tsne_projection.png}
    \caption{t-SNE visualization of CLIP feature space (ViT-L/14). Colors denote SDNET2025 classes. The \emph{fixed} class forms a tight cluster, while \emph{missing} samples exhibit significant overlap with both \emph{loosened} and \emph{fixed}, explaining the lower precision for this minority class.}
    \label{fig:tsne}
\end{figure}

\subsection{Error Analysis}
To provide deployment-oriented diagnostics, we train a balanced logistic probe (150 shots/class) on the released feature bank and plot the resulting confusion matrix in Figure~\ref{fig:confusion}.
The accompanying classification report (\texttt{tables/confusion\_report.txt}) quantifies the recall gap between \emph{missing} and the other two classes, guiding targeted data augmentation.

\begin{figure}[htbp]
    \centering
    \includegraphics[width=0.7\linewidth]{figures/confusion_matrix.png}
    \caption{Confusion matrix of the balanced logistic probe evaluated on the test set. The matrix reveals that \emph{missing} samples are frequently misclassified as \emph{fixed} (low precision), while \emph{loosened} detection suffers from moderate recall issues. Class-specific F1 scores and detailed metrics are provided in \texttt{tables/confusion\_report.txt}.}
    \label{fig:confusion}
\end{figure}

\section{Discussion}
\textbf{Class imbalance} remains the key bottleneck: although overall accuracy improved by 17.98 percentage points, the \emph{missing} class still exhibits precision below 0.30 in some configurations.
Future work should pair our feature bank with synthetic oversampling, focal losses, or active annotation of scarce anomalies.

\textbf{Calibration and deployment}: Temperature scaling hooks in \texttt{train\_final\_improved\_clip.py} can suppress over-confident false negatives---critical for safety inspections where recalls are prioritized.

\textbf{Extending to LoRA fine-tuning}: The repository documents how to introduce \texttt{peft}-based LoRA on upper CLIP layers, which we expect to push accuracy beyond 75\% once compute permits.

\section{Conclusion}
We consolidated a reproducible improvement pipeline for CLIP-based bolt defect recognition on SDNET2025, released accompanying tables and figures, and packaged the narrative into a LaTeX-ready manuscript.
Researchers can reuse the provided assets as-is, swap in new feature banks, or extend the methodology toward adapter tuning and ensemble deployment for field inspections.

\bibliographystyle{ieeetr}
\bibliography{references}

\end{document}

